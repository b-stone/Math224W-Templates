\documentclass{amsart}

%%------------------------------------------------------
%% PLEASE DO NOT EDIT THIS 
%% SECTION UNLESS INSTRUCTED 
%% TO DO SO!

\usepackage[utf8]{inputenc}
\usepackage[text={6in,9in},centering]{geometry} 
\usepackage{enumitem}
\usepackage[doublespacing]{setspace} % Makes the document double spaced.
\usepackage[colorlinks=true]{hyperref}

\setlist[enumerate]{itemsep=3pt,topsep=3pt}
\setlist[enumerate,1]{label=(\alph*)}

\newcommand{\WAtitle}[2]{\noindent\textbf{\Large Writing Assignment \#{#1} \hfill due #2}\\}
\newcommand{\WAauthors}[1]{\noindent {#1}\\}
\newcommand{\defi}[1]{\textbf{\textit{#1}}}


\theoremstyle{definition}
\newtheorem*{defn*}{Definition}
\newtheorem*{prop*}{Proposition}
\newtheorem*{example*}{Example}
\newtheorem*{thm*}{Theorem}


%------------------------------------------------------
% Packages -- If you find new packages you like, use them here. 
%------------------------------------------------------
\usepackage{amssymb, amsmath, latexsym, amsfonts, amsthm, mathrsfs} % Standard packages that are nice to have.


%------------------------------------------------------
% New Macros to make life easier try making your own!
%------------------------------------------------------
\DeclareMathOperator{\Row}{row}
\DeclareMathOperator{\Col}{col}
\DeclareMathOperator{\trace}{Tr}

\newcommand{\ds}{\displaystyle} % no one likes to write \displaystyle, so we change it to \ds


%------------------------------------------------------
% Main Document Body
%------------------------------------------------------
\begin{document}

\WAtitle{2}{February 6, 2019} %% Put the assignment number in the first { } and the due date in the second { }
\WAauthors{Rella Stone and Tipper Gibbons} %% Put the authors here


\begin{defn*}
	A function $f$ mapping from the domain $D$ to a range $R$ is \defi{additive} if $f(\alpha+\beta) = f(\alpha)+f(\beta)$ for any elements $\alpha$, $\beta$ in the domain. 
\end{defn*}

\begin{prop*}[Additivity of Matrices]
	Every $2\times 2$ matrix of real numbers is additive. 
\end{prop*}
\begin{proof}
	
	Let $A = \begin{bmatrix}
		a & b \\
		c & d
	\end{bmatrix}$ be a matrix with $a$, $b$, $c$, $d$ real numbers. If we let $\alpha = \begin{pmatrix}
		\alpha_1 \\
		\alpha_2
	\end{pmatrix}$ and $\beta = \begin{pmatrix}
		\beta_1 \\
		\beta_2
	\end{pmatrix}$ be two vectors in the domain of $A$, then we need to show that $A(\alpha + \beta) = A\alpha + A\beta$. To do this we will show the left hand side of the equation equals the right hand side. 

	On the left hand side, using the matrix as a function, we have 
	\begin{align*}
		A(\alpha + \beta) & = \text{Calculate this} \\
			& =  \\
			& = \text{Don't forget the period}.
	\end{align*}

	Likewise on the right hand side, we obtain
	\begin{align*}
		A\alpha + A\beta & = \text{Calculate this} \\
			& =  \\
			& = \text{Don't forget the period}.
	\end{align*}	

	As $A(\alpha + \beta)$ and $A\alpha + A\beta$ equal the same expression, we know they are equal and thus $A$ is additive. 
\end{proof}


\begin{prop*}[\S1.3, \#28a] 
Let $A$ be an $m \times n$ matrix with a row consisting entirely of zeros. If $B$ is an $n \times p$ matrix, then $AB$ has a row of zeros.
\end{prop*}

\begin{proof}
	Start your proof with the sentence ``Assume that the $i$-th row of $A$ consists entirely of zeros; that is, assume that 
	\[
		\Row_i(A) = \begin{bmatrix} 0 & 0 & \dots & 0\end{bmatrix}.'' 
	\]
	{\bf Do not} assume that $i=1$! {\it Which} row of $AB$ consists entirely of zeros?  {\bf \underline{Why?}}  Use a similar approach for \#28(b). 
\end{proof}






\begin{prop*}[\S1.3, \#28b] 
(The statement you wish to prove should be written here; for example: ``For all square matrices $A$, if $A$ is symmetric, then $A^T$ is symmetric.'')
\end{prop*}

\begin{proof}
	Write your proof here using good proof techniques. 
\end{proof}




%-----------------------------------------------------------------
% Note: Check out WA-0.tex for the proof of \S1.3, \#43c
%-----------------------------------------------------------------

\begin{defn*}
If $A = [a_{ij}]$ is an $n \times n$ matrix, then the \defi{trace} of $A$, $\trace(A)$, is defined to be the sum of the elements on the main diagonal of $A$.  That is, 
\[
\trace(A) = \sum_{i = 1}^n a_{ii}.
\]
\end{defn*}



\begin{prop*}[A Property of the Trace, \S1.3, \#43e] 
If $A$ is an $n \times n$ matrix, then $\trace(A^{T} A) \geq 0$.
\end{prop*}

%-----------------------------------------------------------------
% Feedback to the original proof-writer: You have the right idea,
% but you didn't use the definition of matrix multiplication 
% correctly.  Write down an example of the trace of a 2 by 2 matrix
% A (like row_1(A) = [2,-1], row_2(A) = [-2,0]) and its transpose, 
% then find product, then find the trace of the product (it's 
% bigger than 4).  That will show you where you went wrong in the
% last line of your proof.
%-----------------------------------------------------------------

\begin{proof} 
Assume that $A$ is an $n \times n$ matrix. By definition of trace, $\trace(A^{T}A) = \trace(A^{T})\trace(A)$.  By \#43d (shown below), $\trace(A^{\text{T}}) =\trace(A)$.

Hence, $\trace(A^{T}A) = \left( \sum_{i=1}^n a_{ii} \right)^2$, which is greater than or equal to zero because it is a square.
\end{proof}




\begin{prop*}[More Properties of the Trace, \S1.3, \#43a,b,d]

% Put your statement you want to prove here. 

\end{prop*}





\begin{prop*}[\S1.4, \#4(b)] 
If $A$, $B$, and $C$ are matrices of appropriate sizes, then $(A+B)C = AC + BC$.
\end{prop*}

%-----------------------------------------------------------------
% Construct the proof of this statement by putting the following
% statements in order within a proof environment; then delete this 
% comment.  Note that the expressions in the align* environment 
% should ultimately remain in there, but that you will need to 
% reorder some or them so that they flow naturally (think about the 
% steps you would take to simplify an expression to work toward your
% goal
%-----------------------------------------------------------------

\begin{proof}
This equality holds for all $1 \leq i \leq m$ and $1 \leq k \leq p$, and thus $(A+B)C = AC + BC$ as desired.\textbf{}

By the properties of matrix addition, the matrix $A+B$ has entries $a_{ij} + b_{ij}$ for $1 \leq i \leq m$ and $1 \leq j \leq n$.

By the definition of matrix multiplication and a previous writing assignment (\S1.2 \#17),

Suppose that $A = [a_{ij}]$ and $B = [b_{ij}]$ are $m\times n$ matrices and $C = [c_{jk}]$ is an $n \times p$ matrix.

Let $m$, $n$, and $p$ be positive integers.

\begin{align*}
[(A+B)C]_{ik} 
    &= \Row_{i}(A+B)^T \cdot \Col_{k}(C) \\
    &= \sum_{j = 1}^n (a_{ij} + b_{ij})c_{jk}\\
    &= [AC + BC]_{ik}\\
    &= \Row_{i}(A)^T\cdot \Col_k(C) + \Row_{i}(B)^T \cdot \Col_k(C)\\
    &= [AC]_{ik} + [BC]_{ik}\\
    &= \sum_{j = 1}^n a_{ij}c_{jk} + \sum_{j = 1}^n b_{ij}c_{jk}
\end{align*}
\end{proof}





\begin{thm*}[\S1.4, \#7] 
Prove that $CA = \sum_{j = 1}^m c_j A_j$. 
\end{thm*}

%-----------------------------------------------------------------
% Note: that does *not* look like a well-formed proposition! If I 
% were grading this statement I would ask "What is C? What is A? 
% What about m? Do propositions have words like "Prove that" in 
% them?
%-----------------------------------------------------------------


\begin{proof} 
%%% Your proof here.  Probably a good idea to work it out on paper first.
\end{proof}







	

\end{document}